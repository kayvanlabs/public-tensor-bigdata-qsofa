%should only be 250 words

The quick Sequential Organ Failure Assessment (qSOFA) scoring system is a method to determine which individuals are at risk to progress to poor outcomes related to sepsis using minimal variables. We used Support Vector Machine, Learning Using Concave and Convex Kernels, and Random Forest to predict an increase in qSOFA score using electronic health record (EHR) data, electrocardiograms, and arterial line signals.

A Random Forest model trained on ECG data alone shows an improved performance when tensor decomposition is used to reduce the feature space for predictions in a 6-hour time frame (AUROC 0.67 $\pm$ 0.06 compared to 0.57 $\pm$ 0.08). Adding waveform features from an arterial line signal can further improve performance (AUROC 0.69 $\pm$ 0.07), and the benefits of structuring the data as a tensor and performing tensor decomposition can still be seen (AUROC 0.71 $\pm$ 0.07). Lastly, adding EHR data features to a tensor-reduced signal model further improves performance (AUROC 0.77 $\pm$ 0.06).

Despite a reduction in performance going from an EHR data-informed model to a tensor-reduced waveform data model, the waveform-informed model offers distinct advantages. The first advantage of a signal features-based model is that predictions can be made on a continuous basis in real-time, and second is that these predictions would not be limited by the availability of EHR data. Additionally, structuring the waveform features as a tensor conserves structural and temporal information that would otherwise be lost if the data were presented as flat vectors.