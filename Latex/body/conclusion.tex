% The main conclusions of the study may be presented in a short Conclusions section, which may stand alone or form a subsection of a Discussion or Results and Discussion section.

% jonathan: rewrite to match changes to abstract
In this study, predictions of increase in qSOFA score were created using EHR data and tensor-reduced signal features. It is possible to make a prediction of increase in qSOFA score using EHR data alone (AUROC 0.781 $\pm$ 0.113), and results can be improved if tensor-reduced ECG and arterial line features are added, (AUROC 0.780 $\pm$ 0.112), but results are mixed when signal features are directly added without tensor reduction (AUROC 0.774 $\pm$ 0.115). This may be because the models are overwhelmed with information, whereas tensor reduction improves performance because only pertinent information is given and noise is removed.

Lastly, in the scenario where EHR data is not available, predictions made by using only tensor-reduced ECG features gives a close approximation to the EHR data-only model (AUROC 0.737 $\pm$ 0.123), and the addition of arterial line features improves performance compared to ECG features alone (AUROC 0.753 $\pm$ 0.116). 

The advantage of a signal features-based model is that predictions can be made in the ICU on a continuous basis in real-time, and it would not be limited by the wait times or availability of EHR data variables. From a clinical standpoint, further developing an ECG-only model would be advantageous as it is possible to monitor ECG remotely outside the hospital. Devices such as Holter monitors and Zio patches could be used so that a patient with initially low qSOFA could be monitored at home, with a 6-hour window to predict an increased risk for poor outcomes. Six hours would be adequate time for warning and arrival to the emergency department to seek appropriate treatment.

% jonathan: Think about other strengths and limitations that should be added
We stress that, while it may not achieve F1 or AUROC scores as high as the EHR model, our signal features-only model offers an advantage in that it is not prone to issues such as availability or inaccuracies of EHR data. Furthermore, it is continuously collected allowing for real-time evaluation and assessment. For future work, we recommend (1) the combination of EHR, tensor-reduced ECG, and tensor-reduced arterial line (qSOFA AUROC 0.800) in the hospital or ICU and (2) tensor-reduced  ECG only for home monitoring.