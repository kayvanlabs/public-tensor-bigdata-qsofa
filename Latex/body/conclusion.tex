% The main conclusions of the study may be presented in a short Conclusions section, which may stand alone or form a subsection of a Discussion or Results and Discussion section.

% jonathan: rewrite to match changes to abstract
In this study, predictions of increase in qSOFA score were created using tensor-reduced signal features and EHR data. It is possible to make a prediction of increase in qSOFA score using ECG data alone (for RF, AUROC 0.6690 $\pm$ 0.0644; for LUCCK, 0.6489 $\pm$ 0.0732), and results can be improved if tensor-reduced arterial line features are added, (for RF, AUROC 0.7064 $\pm$ 0.0650; for LUCCK, 0.7073 $\pm$ 0.0700), but results are mixed when signal features are directly added without tensor reduction (for RF, AUROC 0.6855 $\pm$ 0.0650; for LUCCK, 0.6906 $\pm$ 0.0689). This may be because the models are overwhelmed with information, whereas tensor reduction improves performance because only pertinent information is given and noise is removed. 

The previous experiments simulate the scenario when no EHR data is available. When EHR data is available and CP-ALS is used to reduce the feature space of the signal data, results can be further improved (for RF, AUROC 0.7691 $\pm$ 0.0617; for LUCCK, 0.7305 $\pm$ 0.0689). This indicates that ECG signal features, Arterial Line signal features, and EHR data features can all contribute to sepsis prognosis. 

That said, we wish to draw attention to the first scenario, with signals information alone used for model training. The advantage of a signal features-based model is that predictions can be made in the ICU on a continuous basis in real-time; this model would not be limited by the wait times or availability of EHR data variables. From a clinical standpoint, further developing an ECG-only model would be advantageous as, one, it is minimally invasive compared to an arterial line, and two, it is possible to monitor ECG remotely outside the hospital. Devices such as Holter monitors and Zio patches could be used so that a patient with initially low qSOFA could be monitored at home, with a 6-hour window to predict an increased risk for poor outcomes. Six hours would be adequate time for warning and arrival to the emergency department to seek appropriate treatment.

% jonathan: Think about other strengths and limitations that should be added
We stress that, while it may not achieve F1 or AUROC scores as high as the model including EHR data, our signal features-only model offers an advantage in that it is not prone to issues such as availability or inaccuracies of EHR data. Furthermore, it is continuously collected allowing for real-time evaluation and assessment. For future work, we recommend (1) the combination of EHR, tensor-reduced ECG, and tensor-reduced arterial line for use in the hospital or ICU and (2) tensor-reduced ECG only for use in home monitoring.