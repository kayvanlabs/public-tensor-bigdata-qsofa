% This should explore the significance of the results of the work, not repeat them. A combined Results and Discussion section is often appropriate. Avoid extensive citations and discussion of published literature.

RF and LUCCK models performed similarly across different experiments, both performing better than SVM when tensor reduction was applied to the dataset. RF's improvement in performance could be due to its inherent ability to select subsets of features and determine feature importance. (TODO: Add something about LUCCK)

For RF and LUCCK, both F1 Score and AUROC tended to increase when moving from no tensor reduction to tensor reduction. For example, in the 6-hour dataset, mean F1 score increased from (TODO: values) and SD decreased from (TODO: values). A similar increase was observed in mean AUROC (TODO: values) and SD (TODO: values) going from using no tensor reduction to using CP-ALS with rank 2. SVM does not follow this trend, however, and tends to increase in performance as more information is added to the model, with no tensor reduction performing the best.

For 6-hour and 12-hour data, including the arterial line features improved both mean F1 Score and mean AUROC across different CP-ALS ranks, as can be seen comparing Figures \ref{fig:ecgonly} and \ref{fig:sigonly}. For 6-hour data, rank (TODO: Make sure this is still rank 2) may be the best-performing rank as it maintains most information in signal features while also removing redundancy and noise. For 12-hour data, however, increasing rank to 4 improves performance. This suggests that the 12-hour gap requires more information to make a prediction than the 6-hour gap. Performance is fairly stable across rank for all model types, with SVM once again performing best when no tensor reduction is applied. 

Adding EHR data to the signal features, presented in Figure \ref{fig:sigEHR}, does not show a significant improvement for 6-hour data; the increase in mean AUROC that comes from incorporating EHR to the tensor-reduced signal features at rank 2 is (TODO: values), and the decrease in SD is (TODO: values). By comparison, 12-hour data does have increased performance compared to no EHR data. This modest improvement for the 6-hour data may be because most of the information relevant to predicting increase in qSOFA score is already contained in one of the data types, either signal features or EHR data, such that presenting both data types to the model does not add much new information, instead, just redundant information.

While the results of models trained on tensor-reduced signal features show consistent mean AUROC $\geq$ (TODO: value), it is noted that these experiments were trained on data from only one hospital, and that the datasets used do not feature strong racial and ethnic diversity. To ensure the reproducibility and generalizability of these results, it will be necessary to perform similar experiments on a larger and more diverse dataset in future iterations.

