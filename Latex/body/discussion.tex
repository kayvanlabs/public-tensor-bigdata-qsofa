% This should explore the significance of the results of the work, not repeat them. A combined Results and Discussion section is often appropriate. Avoid extensive citations and discussion of published literature.

RF and LUCCK models performed similarly across different experiments, both performing better than SVM when tensor reduction was applied to the dataset. RF's improvement in performance could be due to its inherent ability to select subsets of features and determine feature importance. In turn, if the data are not easily linearly separable, linear SVM may struggle.
% TODO: from Gil "what do you mean by inherent ability? Relate this to 'only pertinent information and noise removal' in conclusion"

For RF and LUCCK, both F1 Score and AUROC tended to increase when moving from no tensor reduction to tensor reduction. For example, for LUCCK in the 6-hour dataset, mean F1 score increased from 0.4313 to 0.4844 with SD remaining similar (0.0612 to 0.0650), while RF's F1 score increased from 0.4080 to 0.4839 with SD decreasing from 0.0570 to 0.0558. We observed a similar increase in mean AUROC for LUCCK (0.6038 $\pm$ 0.0735 to 0.6489 $\pm$ 0.0732) and RF (0.5693 $\pm$ 0.0764 to 0.6690 $\pm$ 0.0644) going from using no tensor reduction to using CP-ALS with rank 4. SVM does not follow this trend, however, and tends to increase in performance as more information is added to the model, with no tensor reduction performing the best.

For 6-hour data, including the arterial line features improved both mean F1 Score and mean AUROC across different CP-ALS ranks, as can be seen comparing Figures \ref{fig:ecgonly} and \ref{fig:sigonly}. For 12-hour data, RF and LUCCK see some increase in performance across the different ranks, but including both Arterial Line and ECG data decreased SVM's performance when no tensor reduction took place. When CP-ALS was used with ranks 1-3 to reduce the feature space for SVM, there is an increase in performance in the ECG + Arterial Line scenario; this suggests that SVM may not be a reliable model for these scenarios.

Adding EHR data to the signal features, presented in Figure \ref{fig:sigEHR}, further improves performance for both the 6- and 12-hour datasets, across all three model types.

While the results of models trained on tensor-reduced signal features show consistent mean AUROC $\geq$ 0.6500 for both LUCCK and RF, it is noted that these experiments were trained on data from only one hospital, the availability of signals led to a small sample pool, and the datasets used do not feature strong racial and ethnic diversity. To ensure the reproducibility and generalizability of these results, it will be necessary to perform similar experiments on a larger and more diverse dataset in future iterations.

