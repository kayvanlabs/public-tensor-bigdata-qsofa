% This should explore the significance of the results of the work, not repeat them. A combined Results and Discussion section is often appropriate. Avoid extensive citations and discussion of published literature.

RF and LUCCK models performed similarly across different experiments, both performing better than SVM when tensor reduction was applied to the dataset. RF's strong performance across different levels of feature reduction could be due to its bagging and bootstrapping procedures, which work to prevent overfitting and ignore noise \autocite{zhang_random_2012, breiman_random_2001}. In its introductory paper, LUCCK was shown to perform well even when trained with few samples of signal data, in part due to its similarity function, which allows for noise or large deviations in some features to not overwhelm the model \autocite{sabeti_learning_2019}. Although SVM is known to perform well when few training samples are available \autocite{gholami_support_2017}, there are also cases where if the data is feature-dense, linear SVM will perform as well as SVM trained with a nonlinear kernel \autocite{hsieh_dual_2008}, as a large number of features can make a dataset linearly separable \autocite{cervantes_comprehensive_2020}. This may be why the non-tensor-reduced datasets tended to have stronger performance than datasets with tensor reduction for SVM.

For RF and LUCCK, both F1 Score and AUROC tended to increase when moving from no tensor reduction to tensor reduction when using only ECG signal data. For example, for LUCCK in the 6-hour dataset, mean F1 score increased from 0.43 to 0.48 with SD remaining similar (0.06 to 0.07), while RF's F1 score increased from 0.41 to 0.48 without a change in SD. We observed a similar increase in mean AUROC for LUCCK (0.60 $\pm$ 0.07 to 0.65 $\pm$ 0.07) and RF (0.57 $\pm$ 0.08 to 0.67 $\pm$ 0.06) going from using no tensor reduction to using CP-ALS with rank 4. SVM does not follow this trend, however, and tends to increase in performance as more information is added to the model, with no tensor reduction performing the best. We see a similar trend in the 12-hour dataset.

For 6-hour data, including the arterial line features improved both mean F1 Score and mean AUROC across different CP-ALS ranks, as can be seen comparing Figures \ref{fig:ecgonly} and \ref{fig:sigonly}. For 12-hour data, RF and LUCCK results are mixed across the different ranks, but including both Arterial Line and ECG data decreased SVM's performance when no tensor reduction took place. When CP-ALS was used with ranks 1-3 to reduce the feature space for SVM, there is an increase in performance in the ECG + Arterial Line scenario; this suggests that SVM may not be a reliable model for these scenarios.

Adding EHR data to the signal features, presented in Figure \ref{fig:sigEHR}, further improves performance for both the 6- and 12-hour datasets, across all three model types.

While the results of models trained on tensor-reduced signal features show consistent mean AUROC $\geq$ 0.65 for both LUCCK and RF, it is noted that these experiments were trained on data from only one hospital, the availability of signals led to a small sample pool, and the datasets used do not feature strong racial and ethnic diversity. To ensure the reproducibility and generalizability of these results, it will be necessary to perform similar experiments on a larger and more diverse dataset in future iterations.

