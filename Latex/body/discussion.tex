% This should explore the significance of the results of the work, not repeat them. A combined Results and Discussion section is often appropriate. Avoid extensive citations and discussion of published literature.

RF and LUCCK models performed similarly when using only EHR data, both performing better than SVM. This improvement in performance could be due to the inherent ability of RF to select subsets of features and determine feature importance. As such, only RF was used for subsequent experiments.

The models trained using only ECG features exhibited an increase in performance when tensor reduction rank 2 or greater was applied. For example, in the 6-hour dataset, mean F1 score increased from 0.487 to 0.553 and SD decreased from 0.150 to 0.145. A similar increase was observed in mean AUROC (0.653 to 0.737) and SD (0.152 to 0.123) going from using no tensor reduction to using CP-ALS with rank 2.

For 6-hour and 12-hour data, including the arterial line features improved both mean F1 Score and mean AUROC across different CP-ALS ranks, as can be seen comparing Figures \ref{fig:rf_ecgonly} and \ref{fig:rf_sigonly}.  Without tensor reduction, the signal features, although informative, were not able to achieve results to the same level as EHR data alone. With tensor reduction, however, particularly at rank 2, signal features have an increase in performance, and are directly compared to the EHR data's models in Supplementary Materials. Rank 2 may be the best-performing rank as it maintains most information in signal features while also removing redundancy and noise. There is a noted dip in performance for 12-hour data at tensor reduction of rank 2 compared to no tensor reduction, but increasing rank to 4 improves performance. This suggests that the 12-hour gap requires more information to make a prediction than the 6-hour gap.

Adding EHR data to the signal features, presented in Figure \ref{fig:rf_sigEHR}, does not show a significant improvement for 6-hour data; the increase in mean AUROC that comes from incorporating EHR to the tensor-reduced signal features at rank 2 is 0.046, and the decrease in SD is 0.004. By comparison, 12-hour data does have increased performance compared to no EHR data. This modest improvement for the 6-hour data may be because most of the information relevant to predicting increase in qSOFA score is already contained in one of the data types, either signal features or EHR data, such that presenting both data types to the model does not add much new information, instead, just redundant information.

While the results of models trained on tensor-reduced signal features show consistent mean AUROC $\geq$ 0.7, it is noted that these experiments were trained on data from only one hospital, and that the datasets used do not feature strong racial and ethnic diversity. To ensure the reproducibility and generalizability of these results, it will be necessary to perform similar experiments on a larger and more diverse dataset in future iterations.

