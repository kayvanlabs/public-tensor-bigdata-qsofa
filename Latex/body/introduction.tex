% State the objectives of the work and provide an adequate background, avoiding a detailed literature survey or a summary of the results.

% What are Sepsis and Septic shock?
Sepsis is a syndrome induced by an existing infection in the body that produces life-threatening organ dysfunction in a chain reaction. The clinical criteria for sepsis include suspected or documented infection and an increase in two or more Sequential Organ Failure Assessment (SOFA) points. Septic shock, a more severe subset, consists of substantially increased abnormalities \autocite{sepsis-3} and higher risk of mortality \autocite{paoli_epidemiology_2018}. It is imperative to risk-stratify patients early in their course in order to appropriately direct critical, but potentially limited, resources and therapies.

Sepsis' heterogeneity complicates its diagnosis and prognosis. Its current definition, based on SOFA score, requires measurement or collection of variables which may not be immediately available. The quick-SOFA (qSOFA) is a screening tool that can be performed at the bedside. It consists of three criteria - Glasgow Coma Scale of $<$ 15 (indicating mental status change), respiratory rate $\geq$ 22 breaths per minute, and systolic blood pressure $\leq$ 100 mmHg - where two of the three must be met \autocite{sepsis-3}. It includes the poorly characterized variable mental status change, but it is a better predictor of organ dysfunction than systemic inflammatory response syndrome (SIRS), which is less sensitive \autocite{sirs_1992, seymour_assessment_2016}. SIRS is the body's response to a stressor such as inflammation, trauma, surgery, or infection, while sepsis is specifically a response to infection; many septic patients have SIRS, but not all patients who meet SIRS criteria have an infection or experience septic organ failure. In comparison to qSOFA, SIRS has four criteria, three of which must be met to positively identify SIRS. These are: respiratory rate $>$ 20 breaths per minute or partial pressure of CO\textsubscript{2} $<$ 32 mmHg; heart rate $>$ 90 beats per minute; white blood cell count $>$ 12,000/microliter or $<$ 4,000/microliter or bands $>$ 10\%; and temperature $>$38$^{\circ}$C or $<$ 36$^{\circ}$C \autocite{sirs_chakraborty_systemic_2022}. For each of these scoring systems, factors such as comorbidities, medication, and age may confound the phenotype in different patient groups.

A system of sepsis detection which is too strict or time-consuming can delay necessary care to patients, and criteria that are too broad can lead to over-treatment or inappropriate use of limited resources. For example, false positive sepsis prognoses can lead to patients receiving unnecessary care and antibiotics, which contribute to antibiotic resistance and emergence of \quotes{superbugs} \autocite{vanepps_reducing_2018, prestinaci_antimicrobial_2015, chokshi_global_2019}. Similarly, qSOFA is not recommended as a single screening tool for diagnosis of sepsis \autocite{evans_surviving_2021}, but it can be used as a method of predicting prolonged ICU stay or in-hospital mortality \autocite{seymour_assessment_2016}. Predicting the trajectory of a patient with suspected infection may be a more efficient use of resources than detecting existing sepsis, and therefore trajectory prediction is the focus of this study. 

% on EHR data
Many models for detecting, monitoring, or predicting outcomes related to sepsis depend on Electronic Health Record (EHR) data, including SOFA score \autocite{sepsis-3}, EPIC's sepsis model \autocite{wong_external_2021}, and others \autocite{nesaragi_correlation_2021, morrill_signature_2019, taylor_prediction_2016}. EHR data can include static variables like demographics information, and dynamic variables such as vital signs or lab values. While useful for determining a patient's status, EHR data are limited by time. Lab values require time for collection and processing, and continuous variables may be updated less than hourly or at irregular intervals. In contrast, physiological readings, such as those generated from electrocardiography, blood pressure monitoring, or pulse oximetry, are collected continuously and at regular intervals. Our study examines the use of continuous physiological signals, namely electrocardiogram (ECG) and arterial line, in outcome prediction related to sepsis.

% on ECG data
ECG signal information has previously been used in the study of risk for sepsis and sepsis progression \autocite{berger_shock_2013, moorman_cardiovascular_2011, nemati_interpretable_2018}. The advantage that continuous monitoring devices like ECG offer over EHR data is real-time, continuous assessment of a patient's status. In addition, ECG is routinely collected in the intensive care unit (ICU), and is minimally invasive. In our analysis, we also include arterial line, as both SOFA and qSOFA use blood pressure to assess the status of a patient's cardiovascular system status \autocite{sepsis-3}.

% Why use tensors?
Given sepsis' complexity and heterogeneity, it is necessary to incorporate multiple variables into a trajectory prediction method. Modeling data as a tensor provides the ability to observe changes in different variables with respect to time and to one another. The prognosis and severity assessment of sepsis rely on a large amount of heterogeneous data, including body temperature, arterial blood pressure, blood culture tests, and molecular assays. Treatment of sepsis does not rely on any individual variable, but on all of these measurements, which vary as a function of time. Because no individual feature is sufficient, integrating data across time and incorporating structure is necessary for improved sepsis prognosis, and therefore can better inform care decisions.
% TODO: from Gil: "explain how tensor analysis achieves these benefits"

% transition to methods section
In this study, we use ECG and arterial line signals to predict an increase in an individual's qSOFA score, where a qSOFA of $\geq$ 2 indicates poor outcomes related to sepsis. The results of signal-trained models are then compared to models trained using both signals and EHR data. This is to (1) predict which individuals are at risk to decompensate to septic shock, experience future organ failure, or other complications related to sepsis, rather than focusing on a sepsis diagnosis, and (2) assess the usefulness of continuous physiological signals in the event that EHR data are unavailable. 